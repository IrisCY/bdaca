% !TEX encoding = UTF-8 Unicode

\documentclass[a4paper,12pt]{report}
\usepackage[natbibapa,nosectionbib,tocbib,numberedbib]{apacite}
\AtBeginDocument{\renewcommand{\bibname}{Literature}}


\usepackage[utf8x]{inputenc}
\usepackage{graphicx}
\usepackage{enumerate}
\usepackage{url}

\usepackage[colorinlistoftodos]{todonotes}

\usepackage{pifont}

\usepackage{lmodern}
\usepackage{listings}
\lstset{
basicstyle=\scriptsize\ttfamily,
columns=flexible,
breaklines=true,
numbers=left,
%stepsize=1,
numberstyle=\tiny,
backgroundcolor=\color[rgb]{0.85,0.90,1}
}


\let\oldquote\quote
\let\endoldquote\endquote
\renewenvironment{quote}{\footnotesize\oldquote}{\endoldquote}



\title{Big Data and Automated Content Analysis\\ Part I+II (12 ECTS)\\~\\Cursusdossier}
\author{dr. Damian Trilling\\~\\Graduate School of Communication\\University of Amsterdam\\~\\d.c.trilling@uva.nl\\www.damiantrilling.net\\@damian0604\\~\\Office: REC-C, 8\textsuperscript{th} floor}
\date{Academic Year 2018/19}


\begin{document}
\maketitle

\tableofcontents


\chapter{Short description of the course}


``Big data'' is a relatively new phenomenon, and refers to data that are more voluminous, but often also more unstructured and dynamic, than traditionally the case. In Communication Science and the Social Sciences more broadly, this in particular concerns research that draws on Internet-based data sources such as social media, large digital archives, and public comments to news and products This emerging field of studies is also called \emph{Computational Social Science} \citep{Lazer2009} or even \emph{Comutational Communication Science} \citep{Shah2015}.

%One of the big challenges is being able to derive information from these data that can be handled meaningfully and economically at the same time.

The course will provide insights in the concepts, challenges and opportunities associated with data so large that traditional research methods (like manual coding) cannot be applied any more and traditional inferential statistics start to loose their meaning. Participants are introduced to strategies and techniques for capturing and analyzing digital data in communication contexts. We will focus on (a) data harvesting, storage, and preprocessing and (b) computer-aided content analysis, including natural language processing (NLP) and computational social science approaches. In particular, we will use advanced machine learning approaches and models like word embeddings.

To participate in this course, students are expected to be interested in learning how to write own programs where off-the-shelf software is not available. Some basic understanding of programming languages is helpful, but not necessary to enter the course. Students without such knowledge are encouraged to follow the (free) online course at \url{https://www.codecademy.com/learn/python} to prepare.

\chapter{Exit qualifications}

The course contributes to the following three exit qualifications of the Research Master in Communication Science: 


\textit{Expertise in empirical research}


	3.	Knowledge and Understanding: Have in-depth knowledge and a thorough understanding of advanced research designs and methods.


	4.	Skills and abilities: Are able, independently and on their own, to set up, conduct, report and interpret advanced academic research.


\textit{Academic abilities and attitudes}


	6.	Attitude: Accept that scientific knowledge is always 'work in progress' and that arguments must be considered and conclusions drawn on the basis of empirical results and valid criticism.


The exit qualifications are elaborated in the following 11 specifications:
3. Knowledge and Understanding: Have in-depth knowledge and a thorough understanding of advanced research designs and methods. 


3.1. Have in-depth knowledge and a thorough understanding of advanced research designs and methods, including their value and limitations.


3.2.	Have in-depth knowledge and a thorough understanding of advanced techniques for data analysis.


4. Skills and abilities: Are able, independently and on their own, to set up, conduct, report and interpret advanced academic research.


4.1	Are able to formulate research questions and hypotheses for advanced empirical studies


4.2	Are able to develop a research plan, choose appropriate and suitable research designs and methods for advanced empirical studies, and justify the underlying choices. 


4.3	Are able to assess the validity and reliability of advanced empirical research, and to judge the scientific and professional value of findings from advanced empirical research.


4.4	Are able to apply advanced empirical research methods.


6. Academic attitudes


6.1 	Regularly asses their own assumptions, strengths and weaknesses critically.


6.2	Accept that scientific knowledge is always 'work in progress' and that something regarded as 'true' may be proven to be false, and vice-versa.


6.3 	Are keen to acquire new knowledge, skills and abilities. 


6.4 	Are willing to share and discuss arguments, results and conclusions, including submitting one's own work to peer review. 


6.5 	Are convinced that academic debates should not be conducted on the basis of rhetorical qualities but that arguments must be considered and conclusions drawn on the basis of empirical results and valid criticism.





\chapter{Testable objectives}

{\footnotesize{
3. Knowledge and Understanding: Have in-depth knowledge and a thorough understanding of advanced research designs and methods. 


3.1. Have in-depth knowledge and a thorough understanding of advanced research designs and methods, including their value and limitations.


3.2.	Have in-depth knowledge and a thorough understanding of advanced techniques for data analysis.

}}

\begin{enumerate}[A]
\item Students can explain the research designs and methods employed in existing research articles on Big Data and automated content analysis.
\item Students can on their own and in own words critically discuss the pros and cons of research designs and methods employed in existing research articles on Big Data and automated content analysis; they can, based on this, give a critical evaluation of the methods and, where relevant, give advice to improve the study in question.
\item Students can identify research methods from computer science and computer linguistics which can be used for research in the domain of communication science; they can explain the principles of these methods and describe the value of these methods for communication science research.4. Skills and abilities: Are able, independently and on their own, to set up, conduct, report and interpret advanced academic research.
\end{enumerate}

{\footnotesize{
4.1	Are able to formulate research questions and hypotheses for advanced empirical studies


4.2	Are able to develop a research plan, choose appropriate and suitable research designs and methods for advanced empirical studies, and justify the underlying choices. 


4.3	Are able to assess the validity and reliability of advanced empirical research, and to judge the scientific and professional value of findings from advanced empirical research.


4.4	Are able to apply advanced empirical research methods.

 }}

\begin{enumerate}[A]
\setcounter{enumi}{3}
\item Students can on their own formulate a research question and hypotheses for own empirical research in the domain of Big Data.
\item Students can on their own chose, execute and report on advanced research methods in the domain of Big Data and automatic content analysis.
\item Students know how to collect data with scrapers, crawlers and APIs; they know how to analyze these data and to this end, they have basic knowledge of the programming language Python and know how to use Python-modules for communication science research.
\end{enumerate}


{\footnotesize{
6. Academic attitudes

6.1 	Regularly asses their own assumptions, strengths and weaknesses critically.


6.2	Accept that scientific knowledge is always 'work in progress' and that something
regarded as 'true' may be proven to be false, and vice-versa.


6.3 	Are keen to acquire new knowledge, skills and abilities. 


6.4 	Are willing to share and discuss arguments, results and conclusions, including submitting one's own work to peer review. 


6.5 	Are convinced that academic debates should not be conducted on the basis of rhetorical qualities but that arguments must be considered and conclusions drawn on the basis of empirical results and valid criticism.

 }}

\begin{enumerate}[A]
\setcounter{enumi}{6}
\item Students can critically discuss  strong and weak points of their own research and suggest improvements.
\item Students participate actively: reading the literature carefully and on time, completing assignments carefully and on time, active participation in discussions, and giving feedback on the work of fellow students give evidence of this.
\end{enumerate}



\chapter{Planning of testing and teaching}

The seminar consists of 28 meetings, two per week. Each week, in the first meeting, the instructor will give short lectures on the key aspects of the week, followed by seminar-style discussions. Theoretical considerations regarding Big Data and Automated Content Analysis are discussed, and techniques for analyzing Big Data are presented. We also discuss examples from the literature, in which these techniques are applied.


The second meetings each week are practicum-meetings, in which the students will apply what the techniques they have learned to own data sets. Here, they can also deepen their understanding of software tools, prepare their projects and get hands-on help. While there are in-class assignments as well as occasional assignments for at home (e.g., completing an online-tutorial to prepare for class), these are not graded.


To complete the course, next to active participation, the students have to successfully complete three summative graded assignments: two mid-term take-home exam and an individual project, in which they derive an empirical question from a theoretical starting point, and then do an Automated Content Analysis to answer the question. See Chapter 7 for details.


\chapter{Literature}


The following schedule gives an overview of the topics covered each week, the obligatory literature that has to be studied each week, and other tasks the students have to complete in preparation of the class.
In particular, the schedule shows which chapter of \cite{Trilling2016} will be dealt with. Note that some basic chapters, which provide the students with the computer skills necessary to use our tools and explain which software to install, have to be read before the course starts.

Next to the obligatory literature, the following books provide the interested student with more and deeper information. They are intended for the advanced reader and might be useful for final individual projects, but are by no means required literature. Bear in mind, though, that the first three books use slightly outdated examples (e.g., Python 2, now-defunct APIs etc.).

\begin{itemize}
\item \citealp{Russel2013}. Gives a lot of examples about how to analyze a variety of online data, including Facebook and Twitter, but going much beyond that.
\item \citealp{Bird2009}. This is the official documentation of the NLTK package that we are using. A newer version of the book can be read for free at \url{http://nltk.org}
\item \citealp{McKinney2012}: Another book with a lot of examples. A PDF of the book can be downloaded for free on \url{http://it-ebooks.info/book/1041/}.
\item \citealp{VanderPlas2016}: A more recent book on numpy, pandas, scikit-learn and more. It can also be read online for free on \url{https://jakevdp.github.io/PythonDataScienceHandbook/}, and the contents are avaibale as Jupyter Notebooks as well \url{https://github.com/jakevdp/PythonDataScienceHandbook}.
\item \citealp{Salganik2017}: Not a book on Python, but on research methods in the digital age. Very readable, and a lots of inspiration and background about techniques covered in our course.
\end{itemize}




\chapter{Specific course timetable}


\section*{Before the course starts: Prepare your computer.}
\textsc{\ding{52} Chapter 1: Preparing your computer}\\
Follow all steps as outlined in Chapter 1.


\section*{PART I: Basics of Python and ACA}

\section*{Week 1: Introduction}
\subsection*{Wednesday, 6--2. Lecture.}
We discuss what Big Data means, how the concept can be understood, what challenges and opportunities arise, and what the implications are for communication science. 

Mandatory readings (in advance): \cite{boyd2012} and \cite{Kitchin2014}. 

Additional literature, not obligatory to read in advance, but very informative: \cite{Mahrt2013}, \cite{Vis2013}, \cite{Trilling2017a}.

%The articles mentioned above discuss the implications of Big Data methods in a very broad way. You should also have a look at some applied articles in the field to get an idea of the type of research that is currently conducted in the field. Good readings are \citealp{Castillo2014,Ellison2013,Conover2012}. You do not have to read all of them in detail, but should get a general understanding of the types of methods that are used in these studies.


\subsection*{Friday, 8--2. No meeting.}
\textsc{\ding{52} Chapter 2: The Linux command line}\\
\textsc{\ding{52} Chapter 3: A language, not a program}\\

Read the two chapters, and make sure you can reproduce the examples on your computer. Write down specific questions you have, so that you can ask them on Monday. It is encouraged to do so in pairs or groups.




\section*{Week 2: Getting started with Python}

\subsection*{Wednesday, 13--2. Lecture.}
\textsc{\ding{52} Chapter 4: The very, very basics of programming in Python}\\
You will get a very gentle introduction to computer programming. During the lecture, you are encouraged to follow the examples on your own laptop.


\subsection*{Friday, 15--2. Lab session.}
\textsc{\ding{52} Appendix A: Exercise 1}\\
We will do our first real steps in Python and do some exercises to get the feeling. 


\section*{Week 3: Data harvesting and storage}
This week is about data sources and their (dis)advantages. 

\subsection*{Wedneday, 20--2. Lecture.}
A conceptual overview of APIs, scrapers, crawlers, RSS-feeds, databases, and different file formats.

Read the article by \cite{Morstatter2013} in advance. It discusses the quality of data provided by the Twitter API. As a practical example for how ``dirty'' input data (i.e., data that for whatever reason does not come in form of a clean, structured data set like a table) can be parsed and preprocessed, have a look at the method section of the article by \cite{Lewis2013}. 


\subsection*{Friday, 22--2. Lab session.}
\textsc{\ding{52} Chapter 5.1--5.4: Retrieving and storing data}\\
We will write a script to collect some data. 




\section*{Week 4: Sentiment analysis.}
Up till now, we have mainly talked about available data and how to acquire them. From now on, we will focus on analyzing them and cover one technique per week. By now, you should also have gotten some idea about your final project.


\subsection*{Wednesday, 27--2. Lecture.}
We start with an overview of different analytical approaches which we will cover in the next weeks, After that, we will focus on the first of these techniques, sentiment analysis.

Mandatory readings (in advance): \cite{GonzalezBailon2015} and \cite{Hutto2014}.

Suggestions for additional readings:
\begin{itemize}
	\item Examples of (simple) sentiment analyses: \cite{Huang2007,Pestian2012, Mostafa2013}. 
	\item If you want to have a look under the hood of another popular sentiment analysis algorithm, you can read \cite{Thelwall2012}.
\end{itemize}




\subsection*{Friday, 1--3. Lab session.}
\textsc{\ding{52} Chapter 6: Sentiment analysis}\\
You will write a script to read data and conduct a sentiment analysis.






\section*{Week 5: Automated content analysis with NLP and regular expressions.}
Text as written by humans usually is pretty messy. You will learn how to process text to make it suitable for further analysis by using techniques of Natural Language Processing (NLP), and how to extract meaningful information (discarding the rest) using regular expressions. You will also make a first aquintance with the packages NLTK and spacy.




\subsection*{Wedneday, 6--3. Lecture with exercises.}
\textsc{\ding{52} Chapter 7: Automated content analysis}\\
This lecture will introduce you to techniques and concepts like stemming, stopword removal, n-grams, word counts and word co-occurrances, and regular expressions. We will do some exercises during the lecture.

Preparation: Mandatory reading: \cite{Boumans2016}. 




\subsection*{Friday, 8--3. Lab session.}
You will combine the techiques discussed on Wednesday and write a full automated content analysis script using a top-down dictioary or regular-expression approach.



\subsection*{Take-home exam}
In week 5, the first midterm take-home exam is distributed after the Friday meeting. The answer sheets and all files have to be handed in no later than the day before the next meeting, i.e. Tuesday evening (12--5, 23.59).




\section*{Week 6: Web scraping and parsing}

\subsection*{Wednesday, 13--3. Lecture.}
We will explore techniques to download data from web pages and to extract meaningful information like the text (or a photo, or a headline, or the author) from an article on \url{http://nu.nl}, a review (or a price, or a link) from \url{http://kieskeurig.nl}, or similar. 

\subsection*{Friday, 15--3. Lab session.}
\textsc{\ding{52} Chapter 8: Web scraping}\\
We will exercise with web scraping and parsing.





\section*{Week 7: Statistics with Python}

\subsection*{Wednesday, 20--3. Short lecture plus lab session.}
\textsc{\ding{52} Section 3.5: Jupyter Notebook}\\
\textsc{\ding{52} Chapter 12: Statistics with Python}\\
You have worked hard so far, so we'll do something fun and relaxing (of course, fun might be a relative concept in this course\ldots). You are going to learn how to create visualizations, do conventional statistical tests, manage datasets with Python, save the results together with your code and your own explanations -- and all of this within your browser.



\subsection*{Friday, 22--3.  Short lecture plus lab session.}
We will learn how to do data wrangling with pandas: converting between wide and long formats (melting and pivoting), aggregating data, joining datasets, and so on.


\section*{-- Break between block 1 and 2 -- }

\section*{PART II: Advanced analyses}


\section*{Week 8: Dealing with temporal data}

\subsection*{Wednesday, 3--4. Guest lecture by Rens Vliegenthart.}
We will talk about time series analysis. Many data you can collect online have some temporal component in them: think of references to political parties or topics over time, or of the coverage of an organization. The same holds true for non-media data, such as stock exchange rates or unemployment statistics. We will discuss statistical models to analyse such data.


Mandatory reading (in advance): \cite{Vliegenthart2014} and \cite{Strycharz2018}.


\subsection*{Friday, 5--4. Lab session.}
We will use an example to explore how to implement the techniques discussed on Wednesday using statsmodels \citep{statsmodels}.



\section*{Week 9: Supervised Machine Learning 1}
In weeks 9 and 10, you will learn how to work with scikit-learn \citep{scikit-learn}, one of the most well-known machine learning libraries.


\subsection*{Wednesday, 10--4. Lecture.}
We will learn the principles of supervised machine learning and discuss how logistic regression and Naive Bayes classifiers can be used to predict, for instance, movie ratings or topics of news articles. We will also discuss basics evaluation metrics like precision and recall.

Mandatory reading (in advance): \cite{burscher2014}. 

\subsection*{Friday, 12--4. Lab session.}
\textsc{\ding{52} Chapter 10: Supervised machine learning}\\
You will build your first machine learning classifier.



\section*{Week 10: Supervised Machine Learning 2}

\subsection*{Wednesday, 17--4. Lecture with practical exercise.}
We will discuss more in detail how to select the best model for your purpose. We will talk about cross-validation, parameter tuning, and building a pipeline.

\subsection*{Friday, 19--4. No meeting (Easter)}




\section*{Week 11: Unsupervised Machine Learning 1}

\subsection*{Wednesday, 24--4. Lecture.}
We will discuss the basics of unsupervised machine learning, using techniques such as principal component analysis, k-means clustering, and hiearchical clustering.

Mandatory reading (in advance): \cite{burscher2016}.

\subsection*{Friday, 26--4. Lab session.}
You will apply the techniques discussed on Wednesday.


\subsection*{Take-home exam}
In week 11, the second midterm take-home exam is distributed after the Friday meeting. The answer sheets and all files have to be handed in no later than the day before the next meeting, i.e. Tuesday evening (30--4, 23.59).






\section*{Week 12: Unsupervised Machine Learning 2}


\subsection*{Wednesday, 1--5}
We will discuss one of the most popular unsupervised techniques in automated content analysis: topic modeling. In particular, we will focus on LDA.

Mandatory readings (in advance): \cite{Maier2018a} and \cite{Tsur2015}. 

\subsection*{Friday, 3--5}
\textsc{\ding{52} Chapter 11: Unsupervised machine learning}\\
You will apply the techniques discussed on Wednesday using gensim \citep{Rehurek2010}.



\section*{Week 13: Word embeddings}

\subsection*{Wednesday, 8--5}
In this week, we will talk about a problem of standard forms of ACA: they treat words as independent from each other, and as either present or absent. For instance, if ``teacher'' is a feature in a specific model, and a text mentions ``instructor'', then this is not captured -- even though it probably should matter, at least to some extend. Word embeddings are a technique to overcome this problem. But also, they can reveal hidden biases in the texts they are trained on.

Mandatory readings (in advance): \cite{Kusner2015} and \cite{Garg2017}



\subsection*{Friday, 10--5}
We will apply a word2vec model.



\section*{Week 14: Wrapping up \& moving on}

\subsection*{Wednesday, 15--5. Lecture}
In this meeting, we will wrap up what has been covered in this course and discuss what other techniques and approaches exist that we did not have time to cover in detail, such as deep learning. 

\subsection*{Friday, 17--5. Open Lab}
Possibility to ask last questions regarding the final project.

\subsection*{Final project}
Deadline for handing in: Wednesday, 29--5, 23.59.


\chapter{Testing}
An overview of the testing is given in Table \ref{testmatrix}.

\begin{table}[]
\footnotesize{
\centering
\caption{Test matrix}
\label{testmatrix}
\begin{tabular}{p{8cm}p{2cm}p{2cm}p{2cm}}
                                                                                                                                                                                                                                                                                                                                                                                            & In-class assignments, reviewing work of fellow students, active participation  
& Mid-term take home exams 
& Final individual project \\
                                                                                                                                                                                                                                                                                                                                                                                                                   &  (precondition)                                                                                            & ($2 \times 20$\% of final grade)   & (60\% of final grade)    \\
A. Students can explain the research designs and methods employed in existing research articles on Big Data and automated content analysis.                                                                                                                                                                                                                      & X                                                                                            & X                       &                          \\
B. Students can on their own and in own words critically discuss the pros and cons of research designs and methods employed in existing research articles on Big Data and automated content analysis; they can, based on this, give a critical evaluation of the methods and, where relevant, give advice to improve the study in question.
& X                                                                                            & X                       &                          \\
C. Students can identify research methods from computer science and computer linguistics which can be used for research in the domain of communication science; they can explain the principles of these methods and describe the value of these methods for communication science research.4. Skills and abilities: Are able, independently and on their own, to set up, conduct, report and interpret advanced academic research.
                                                                & X                                                                                            & X                       & X                        \\
D. Students can on their own formulate a research question and hypotheses for own empirical research in the domain of Big Data.                                                                                                                                                                                                     &                                                                                              &                         & X                        \\
E. Students can on their own chose, execute and report on advanced research methods in the domain of Big Data and automatic content analysis.                                                                                                                                                                                             &                                                                                              &                         & X                        \\
F. Students know how to collect data with scrapers, crawlers and APIs; they know how to analyze these data and to this end, they have basic knowledge of the programming language Python and know how to use Python-modules for communication science research.                                                                         & X                                                                                            & X                       & X                        \\
G. Students can critically discuss  strong and weak points of their own research and suggest improvements.                                                                                                                                                                                                                                         &                                                                                              &                         & X                        \\
H. Students participate actively: reading the literature carefully and on time, completing assignments carefully and on time, active participation in discussions, and giving feedback on the work of fellow students give evidence of this.                                                                                                                                      & X                                                                                            &                         &                         
\end{tabular}
}
\end{table}




\section*{Grading}

The final grade of this course will be composed of the grade of two mid-term take home exams ($2 \times 20$\%) and one individual project (60\%).

\subsection*{Mid-term take-home exam ($2 \times 20$\%}
In two mid-term take-home exam, students will show their understanding of the literature and prove they have gained new insights during the lecture/seminar meetings. They will be asked to critically assess various approaches to Big Data analysis and make own suggestions for research. Additionally, they need to (partly) write the code to accomplish this.

\subsection*{Final individual project (60\%)}
The final individual project typically consists of the following elements:
\begin{itemize}
\item introduction including references to relevant (course) literature, an overarching research question plus subquestions and/or hypotheses (1–2 pages);
\item an overview of the analytic strategy, referring to relevant methods learned in this course;
\item carefully collected and relevant dataset of non-trivial size;
\item a set of scripts for collecting, preprocessing, and analyzing the data. The scripts should be well-documented and tailored to the specific needs of the own project;
\item output files;
\item a well-substantiated conclusion with an answer to the RQ and directions for future research.
\end{itemize}

\subsection*{Grading and 2\textsuperscript{nd} try}
Students have to get a pass (5.5 or higher) for both mid-term take-home exams and the individual project. If the grade of one of these is lower, an improved version can be handed in within one week after the grade is communicated to the student. If the improved version still is graded lower than 5.5, the course cannot be completed. Improved versions of the final individual project cannot be graded higher than 6.0. 


\chapter{Lecturers' team, including division of responsibilities}
dr. Damian Trilling


\chapter{Calculation of students' study load (in hours)}
\begin{itemize}

\item Elective total: 12 ECTS = 336 hours
\item Reading: 
\begin{itemize}
\item 16 articles, average 20 pages: 320 pages. 6 pages per hour, thus 53 hours for the literature
\item Reading and doing tutorials: 80 hours for reading tutorials to acquire skills.
\item Reading book: 20 hours
\item Reading/preparation total: 153 hours.
\end{itemize}
\item Presence: \\28*2 hours: 56 hours.
\item Mid-term take-home exam, including preparation (2 exams) $2 \times 14$ hours: 28 hours
\item Final individual project, including data collection, analysis, write up: 90 hours
\end{itemize}

Total: 337 hours



\chapter{Calculation of lecturers' teaching load (in hours)}
\begin{itemize}
\item Presence: 56 hours (= 28 * 2 hours)
\item Preparation of weekly lectures, 14 * 4 hours: 56 hours
\item Preparation of weekly tutorials, 14 * 4 hours: 56 hours
\item Assisting students with setting up Virtual Machine, individual help: 20 hours
\item Feedback and grading take-home exams: 25x20 minutes x 2 exams: 17 hours
\item Feedback and grading final projects, including feedback on proposal and individual counseling: 25* 60 min: 25 hours 
\item Administration, e-mails, individual appointments: 10 hours
\end{itemize}
Total: 240 hours

 
 
\bibliographystyle{apacite}
\bibliography{../../bdaca}

 
 
 
\end{document}
